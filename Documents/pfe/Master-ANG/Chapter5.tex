\chapter{Chapter 5}
\section{Introduction}
In this Chapter, we will investigate and try to solve another limitation of the DCA which is its sensitivity to the input class data order.  As previously mentioned, the standard DCA had to be applied to ordered data sets where all class 1 are followed by all class 2. This is to ensure high and satisfactory classification results.  While investigating this DCA limitation, we have noticed that this problem could be linked to the DCA context assessment phase. Therefore, we propose in this Chapter to develop an automated fuzzy dendritic cell algorithm characterized by its stability as a binary classifier.  

This Chapter is structured as follows: In Section 5.2, we present the motivation behind the development of our proposed solution. In Section 5.3, fuzzy clustering techniques which are needed to build our new fuzzy DCA  are detailed. Section 5.4  describes our new developed   algorithm. Experiments are outlined in Section 5.5 and the final Section includes the discussion of the obtained results.
\section{Problem Statement}
A study focusing on the DCA behavior stated that the algorithm gives satisfactory and interesting  classification results when it is, only, applied to databases having ordered  classes \cite{dDCA}. In this Chapter, we try to study carefully  the DCA algorithmic phases while trying to figure out the reasons of such a restriction.

\section{Conclusion}
In this Chapter, we have analyzed the behavior of the standard DCA while trying to overcome its shortcoming as it is sensitive to the class data order.  This investigation led us to the development of the  FCDCM algorithm. FCDCM   presents a fuzzy version of the DCA and it aims at coping with the crisp separation between the two DCs contexts. In order to fix the parameters of the algorithm, we have used various fuzzy clustering techniques, compared them and chosen the most appropriate one. The experimental results demonstrate that our proposed method with the Gustafson-Kessel algorithm ($FCDCM_{GK}$) achieves better results   compared to other methods. 

In the next Chapter, we will try to investigate more the causes of the dendritic cell algorithm sensitivity to the input class data order. We will focus, mainly, on the dendritic cell algorithm detection phase   while suggesting new automated and more adequate solutions for the DCA.




