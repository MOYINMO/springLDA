\chapter{General context}
\section{Introduction}
During this introductory chapter, we will talk about the Problematic and we will try to define the challenges and constraints of our project.Later, we will explain our objectives in order to accomplish our mission. Finally, we will end this section with a presentation of the framework of our internship.
\section{Problematic}
The concept of “smart city” is evolving as a new approach to mitigate and remedy current urban problems and make urban development more sustainable \cite{Alawadhi2012}.in this case, we must have citizen with operational feedback.\\
We focus on twitter microblog to extract  information.This platform is accessible through websites or cellphone applications allowing users to instantly post relevant information about what they are seeing, hearing and experiencing around them. In a SDG case, such platform provide valuable information shared voluntary to inform or alert the society .\\
Information retrieval from microblogs is hindered by many challenges such as: streaming data analysis,the variety of information format and language processing, large datasets, and extracting relevant and fresh information from a huge amount of outdated and redundant data.
\section{Objectives}
The objective of the proposed research is to provide insights on the online interactions and revealed perceptions of stakeholders about the SDG related to cities. We will provide cities with new informational tools, supporting their strategy in order to achieve the SDG11
(sustainable cities and communities) goal and its targets. 
\\
These informational tools will enable to bridge the gap between the vision of the diverse stakeholders, from United Nations, government and non-government agencies, to the civil society and the private sector. Specifically, the proposal will analyze the data posted on the Twitter platform to provide agencies and citizens with operational feedback on how SDGs are perceived and implemented.
\section{framework of our internship}
The internship is on the context of obtaining of the degree of Diploma in Computer Science and Multimedia Engineering.
\\
The internship was proposed by CeRCI Unit. It have an overall duration of four months. 
 \begin{itemize}
 \item Phase of documentation and bibliographic study which have a duration of one month.
 \item Phase of development and experimentation which was concertized by a practical internship in the laboratory CeRCI at  Laval University with two months of work.
 \item Phase of The synthesis and evaluation of the  obtained results which was elaborated in one month.
 \end{itemize} 
\subsection{laboratory CeRCI}
Laboratory CeRCI,Intelligent Communities Research Center,was created in XXXX in Laval University.
\begin{figure} [H]
\begin{center}
\includegraphics [width=2.5cm]{logo.png}
\caption{Logo Laval University}
\end{center}
\end{figure}
\textbf{Organizational chart}

Laboratory CeRCI was founded by five professors: Sehl Mellouli, Monia Rekik, Adnène Hajji, Jacqueline Corbett et Karim Ben Boubaker.
The research activities is centered  to three axes: citizen, governance and technology.  
\section{Conclusion}
During this chapter, we have presented the project, the framework of internship and the work obligations. In the next chapter,we will present and compare the commonly used approach  of information retrieval.

