\chapter{Chapter 7}
\section{Introduction}
The developed non-sensitive COID-$FCDCM_{GK}$ algorithm has shown promising results in terms of classification accuracy. It is based on a deep investigation of its algorithmic steps. Yet, it is important to mention that the COID-$FCDCM_{GK}$ data pre-processing phase    is based on the use of the principal component analysis technique. In \cite{RCDCA, RSTDCA}, we have criticized the fact of using PCA and proposed adequate solutions. Therefore, in this Chapter, we aim to develop an overall automated fuzzy DCA  classifier characterized by its non-sensitivity to the input class data order and based on a robust data pre-processing phase.  

This Chapter is organized as follows: Section 7.2 describes  an analysis of the COID-$FCDCM_{GK}$ algorithm. Section 7.3 details the hybrid rough automated fuzzy DCAs. These classifiers are non-sensitive automated algorithms based on the use of the traditional  rough set theory for data pre-processing. Section 7.4 presents the hybrid fuzzy-rough automated fuzzy DCAs. These algorithms are non-sensitive automated classifiers based on the use of the theory of fuzzy rough sets for data pre-processing. Section 7.5 proposes  our newly algorithm. Finally,  Section 7.6 gives a summary of the whole work.

\section{Summary}
In this Section, a summary of the whole work is presented.....
\section{Conclusion}
In this Chapter, we have proposed a hybrid automated and maintained fuzzy dendritic cell immune classifier.  COID-FLA-$FCDCM_{GK}$ can be seen as the output of several investigations conducted on the standard DCA version. A set of hypotheses has been checked and based on the results drawn for the six hypotheses, the COID-FLA-$FCDCM_{GK}$ was developed. The worthy characteristics of the latter algorithm were highlighted in this Chapter as well.





