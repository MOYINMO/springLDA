\chapter{Chapter 6}
\section{Introduction}
Still focusing on the DCA sensitivity to the input class data order, we have noticed that this   limitation cannot only be linked to the DCA context assessment phase, as presented in the previous Chapter, but also  to the DCA detection phase. We hypothesize that there is a second possible cause related to the DCA sensitivity.  It is possible that the induced signal base generated by the DCA detection phase  contains disagreeable objects such as noisy, incoherent or redundant instances. Such objects may influence the DCA behavior and, thus, maintaining the   signal base looks essential. To achieve this, we propose in this Chapter  to expand a second fuzzy DCA version based on a maintenance technique. 

This Chapter is structured as follows: In Section 6.2, we give an overview of the maintenance database policies while in Section 6.3, we present  our new fuzzy maintained DCA. The experiments and the obtained results are outlined in Section 6.4.
\section{Overview of the Maintenance Database Policies}
The quality of the database is very important to generate accurate results and to have the possibility to learn models from the presented data.  To achieve this, the disagreeable objects - especially noisy and redundant instances which affect negatively the quality of the results - have to be eliminated from the data set. To address this situation, data maintenance is a feasible way. The goal of database maintenance methods is to select the most representative information from a given data set. 

Many works dealing with database maintenance have been proposed in literature \cite{D4}. Most of them are based on updating a database by adding or deleting instances to optimize and reduce the initial database. These policies include different operations such as:  the outdated, redundant or inconsistent instances may be deleted, groups of objects may be merged to eliminate redundancy and improve reasoning power, objects may be re-described to repair incoherencies, signs of corruption in the database have to be checked and any abnormalities in the database which might signal a problem has to be controlled. A database which is not maintained can become sluggish and without accurate data users will make uninformed decisions. The database maintenance policies may be categorized into two main heads; the selective reduction approaches and the deletion reduction approaches.



\section{Conclusion}
In this Chapter, we have presented  a new fuzzy hybrid evolutionary algorithm, COID-$FCDCM_{GK}$, aiming at solving the DCA sensitivity to input class data order. COID-$FCDCM_{GK}$  is based on two hypotheses which were studied and checked. COID-$FCDCM_{GK}$ is a stable classifier which outperforms not only the state-of-the-art classifiers but also the standard version of the DCA as well as the DCA hybrid proposed versions.  

Up to now, we have made several investigations on the dendritic cell functioning as well as its algorithmic steps. While investigating the DCA detection and context assessment phases, we have proposed solutions to overcome the DCA sensitivity to the input class data order. While investigating the DCA data pre-processing phase, we have proposed new rough and fuzzy-rough methods leading to a more robust DCA classifiers. Note that the proposed rough and fuzzy-rough DCAs seen in the previous Chapters  are applied to ordered data sets only as they are sensitive to the input class data order. Therefore and at the end of this dissertation, we will propose a more general   fuzzy-rough, maintained  and  non-sensitive dendritic cell algorithm. This will be dealt with in the next Chapter.
